\documentclass{article}

\usepackage[utf8]{inputenc}
\usepackage[finnish]{babel}
\usepackage{hyperref}
\usepackage{amsmath}

\newcommand{\BigO}[1]{\ensuremath{\operatorname{O}\bigl(#1\bigr)}}

\setlength{\parindent}{0.0in}
\setlength{\parskip}{0.1in}

\begin{document}
\title{Tiralabra 2013 periodi III \\ Viikkoraportti V}
\author{Mika Viinamäki}
\maketitle

\section{Viikon saavutukset}

Tästä viikosta on kovin vähän kerrottavaa --- aloitin testaus- ja toteutusdokumentaatioiden kirjoittamisen, mutta molemmat ovat vielä selkeästi vaiheessa. Huomasin tosin että \LaTeX illa saa \texttt{pgfplots}-paketin avulla kivoja käppyröitä aikaan!

Lisäksi sain lisää ymmärrystä siitä miten kehno pakkaustoteutukseni oikeastaan onkaan --- tapa millä käytän hajautustaulua on melkoinen muistisyöppö ja tavallaan tuhoaa pakkausnopeuden. Menee sitten toteutusdokumentaation puutteisiin, en taida enää tässä vaiheessa lähteä tekemään suuria muutoksia ohjelmaan. Myös joitakin koodin rakenteeseen liittyviä puutteita tuli huomattua.

\section{Seuraavaksi}

\begin{itemize}
    \item Lisää dokumentteja
\end{itemize}

\end{document}
