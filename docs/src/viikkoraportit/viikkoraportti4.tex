\documentclass{article}

\usepackage[utf8]{inputenc}
\usepackage[finnish]{babel}
\usepackage{hyperref}
\usepackage{amsmath}

\newcommand{\BigO}[1]{\ensuremath{\operatorname{O}\bigl(#1\bigr)}}

\setlength{\parindent}{0.0in}
\setlength{\parskip}{0.1in}

\begin{document}
\title{Tiralabra 2013 periodi III \\ Viikkoraportti IV}
\author{Mika Viinamäki}
\maketitle

\section{Viikon saavutukset}

Tällä viikolla sain aikaan kaikenlaisia benchmarkkeja --- sellaisia löytyy nyt pakkausnopeudesta, pakkaustehosta ja hajautustaulun suorituskyvystä. Joskin viimeisin ei ole toistaiseksi mitenkään erikoinen. Käytin jo viimeksi mainittua Caliper-kirjastoa aikatesteissä, pakkaustehoa testaa vain pieni yksinkertainen koodinpätkä. Tuloksena näistä kaikista on mielenkiintoisia numeroita, joista varmaan riittää analysoitavaa. Pieni varoituksen sana: varaa aikaa, jos kokeilet pakkausnopeuden benchmarkia --- itselläni sen ajaminen kestää noin 40 minuuttia.

Lisäksi rakensin itse pääohjelman. Otin vapauden käyttää Apache CLI -kirjastoa komentoriviparametrien käsittelyn apuna --- sitä ei ole hirveän siistiä tehdä käsin ja se tuskin on hirveän olennaista tämän kurssin kannalta.

Kaikkien tietorakenteiden ja taulukon käsittelyn apumetodien pitäisi nyt olla omia. Testeissä saattaa tosin olla jotain muuta kuin omaa jäljellä, mutta tarviiko niiden testeissä täysin omia ollakaan?

Ohjelma noin muuten on periaatteessa valmis, mitä nyt optimointia ja virittelyä voisi tehdä, mutta keskittynen seuraavaksi kurssin erinäisten dokumenttien kirjoittamiseen.

\section{Seuraavaksi}

\begin{itemize}
    \item Dokumentaatioiden kirjoittelua
\end{itemize}

\end{document}
