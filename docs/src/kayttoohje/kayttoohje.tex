\documentclass{article}

\usepackage{color}
\usepackage[utf8x]{inputenc}
\usepackage[finnish]{babel}
\usepackage{hyperref}
\usepackage{spverbatim}

\newcommand{\BigO}[1]{\ensuremath{\operatorname{O}\bigl(#1\bigr)}}

\setlength{\parindent}{0.0in}
\setlength{\parskip}{0.1in}

\begin{document}
\title{Tiralabra 2013 periodi III \\ Käyttöohje}
\author{Mika Viinamäki}
\maketitle

\section{Ohjelman ajaminen}

Paketin juuressa on tiedosto \verb+lzw.jar+, jonka sisältää ohjelman ja sen riippuvuudet. Ohjelmaa voi ajaa terminaalissa (ainakin Linuxilla) seuraavasti:

\begin{verbatim}
$ java -jar lzw.jar
\end{verbatim}

Ohjelman kutsuminen ilman parametreja --- siis kuten yllä --- tulostaa lyhyen käyttöohjeen.

\begin{verbatim}
usage: lzw
 -D,--decode-chunks          Decode data that has been encoded using
                             --encode-chunks parameter.
 -d,--decode                 Decode data from STDIN.
 -E,--encode-chunks <SIZE>   Encode data with maximum LZW dictionary size.
                             Default size is 10000.
 -e,--encode                 Encode data from STDIN.
 -h,--help                   Show this message.
\end{verbatim}

\section{Tiedon pakkaaminen ja purkaminen}
Ohjelma kykenee pakkaamaan mitä tahansa binäärimuotoista dataa ja purkamaan sillä pakatun datan.

Lyhyitä merkkijonoja voi pakata seuraavasti putkea (pipe) käyttämällä:
\begin{verbatim}
$ echo "Hei maailma" | java -jar lzw.jar -e
pakattu data tulostettuna konsoliin
\end{verbatim}

\begin{samepage}
Pakatun datan voi ohjata halutessaan myös tiedostoon:
\begin{verbatim}
$ echo "Hei maailma" | java -jar lzw.jar -e > pakattu_maailma.lzw
$ cat pakattu_maailma.lzw
pakattu data tulostettuna konsoliin
\end{verbatim}
\end{samepage}

Paketissa on mukana lisäksi muutama pakkauksen testaamiseen soveltuva tiedosto, joita voi pakata seuraavasti:
\begin{verbatim}
$ cat test_files/seitseman_veljesta.txt | java -jar lzw.jar -e >
  seitseman_veljesta_pakattu.lzw
\end{verbatim}

Pakatun tiedoston voi purkaa näin:
\begin{verbatim}
$ cat seitseman_veljesta_pakattu.lzw | java -jar lzw.jar -d
<purettu data tulostettuna konsoliin>
\end{verbatim}

Pakkaamisen ja purkamisen voi lisäksi tehdä yhtäaikaa --- näppärä tapa varmistua siitä, että pakkaus toimii:
\begin{verbatim}
$ cat test_files/seitseman_veljesta.txt | java -jar lzw.jar -e |
  java -jar lzw.jar -d
\end{verbatim}

\subsection{Lohkoissa pakkaaminen}
Datan pakkaamista lohkoissa (selitetty tarkemmin toteutusdokumentissa) voi käyttää antamalla ohjelmalle parametrin \verb+-E+ parametrin \verb+-e+ sijaan. Lohkotilassa pakatut tiedostot pitää purkaa parametrilla \verb+-D+.

Sanakirjan maksimikoon voi antaa parametrina näin:
\begin{verbatim}
$ java -jar lzw.jar -E 1000
\end{verbatim}

\end{document}
